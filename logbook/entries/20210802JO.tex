



Households in the CEX are interviewed once every three months about their expenditures in the previous three months. 
In order to survey a subset of households every month, households are put on one of three different interview 
schedules. In the overall CEX data, households are distributed relatively evenly across the three schedules.
 If the timing of check recipient is truly random, then we should also expect a roughly even distribution of
  interview schedules for each rebate cohort. However, this is not what we find.

Table \ref{tab:interview_randomnessPB} shows the distribution of interview schedules by rebate cohort. There seems to be
 a larger amount of households in each cohort when households have an interview in the month after they received 
 their stimulus check. For example, 42 percent of the May cohort is on the Mar-Jun-Sep-Dec interview schedule.
  Households on this schedule are interviewed in June about their spending in the three months prior to June. 
  It just so happens that households that `` receive " their check in May are more likely to have been interviewed
   in June. This is true for the June and July cohorts as well and this relationship still holds when households
    that receive their check via EFT are removed. Are households more likely to report that they receive their 
    stimulus check the month prior when asked by the interviewer?

\begin{landscape}
    \begin{table}[!t] \centering \sisetup{table-format=3.2} \def\sym X 1{\ifmmode^{ X 1}\else\(^{ X 1}\)\fi}
\caption{Distribution of CEX Interview Schedule}
\begin{tabularx}{\hsize}{@{\hskip\tabcolsep\extracolsep\fill}l*{4}{S}}
\\
\hline\hline
  &  \multicolumn{4}{c}{Panel A: EFT and Check Recipients }  \\  & \multicolumn{1}{c}{Overall CEX}   & \multicolumn{1}{c}{May Cohort}  & \multicolumn{1}{c}{June Cohort}  & \multicolumn{1}{c}{July Cohort} 
\\
\hline
\multicolumn{1}{l}{Interview Schedule} & \\
\\
Jan-Apr-Jul-Oct     &        33\%&        32\%&        35\%&        26\%\\
Feb-May-Aug-Nov     &        34\%&        29\%&        37\%&        39\%\\
Mar-Jun-Sep-Dec     &        34\%&        39\%&        28\%&        34\%\\
\hline
\hline
\\
\hline\hline
  &  \multicolumn{4}{c}{Panel B: Check Recipients Only }  \\  &   & \multicolumn{1}{c}{May Cohort}  & \multicolumn{1}{c}{June Cohort}  & \multicolumn{1}{c}{July Cohort} 
\\
\hline
\multicolumn{1}{l}{Interview Schedule} & \\
\\
Jan-Apr-Jul-Oct     &            &        30\%&        36\%&        28\%\\
Feb-May-Aug-Nov     &            &        34\%&        35\%&        40\%\\
Mar-Jun-Sep-Dec     &            &        36\%&        28\%&        32\%\\
\hline
\hline \end{tabularx} \begin{minipage}{\hsize} \rule{0pt}{9pt} \footnotesize Notes: Data in column 1 come from the entire CEX Sample 2007-2009. Data in columns 2-4 come from our subsample.    \end{minipage} \label{tab:interview_randomnessPB} \end{table}

\end{landscape}