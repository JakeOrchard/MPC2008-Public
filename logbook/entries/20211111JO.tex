

One of Parker's main critiques of Valerie's discussion was that using the National Income and Product Accounts (NIPA)
 to analyse changes in aggregate spending over a small period is problematic; the PCE comes from multiple 
 retail sources and smooths the data [After re-reading Parker's comments, I think the fact that Valerie found 
 changes in the savings rate in the smoothed NIPA data actually supports her conclusion]. 
This in progress section compares the NIPA time series with similar series in the CEX and Retail data (TBD). 


\subsection{How is the PCE Constructed?}

Jake: Will write this section. 


\subsection{NIPA versus CEX}





How does spending in the National Income and Product Accounts (NIPA) compare with spending in the CEX? One key series from 
NIPA is Personal Consumption Expenditures (PCE). The PCE includes all consumption spending by and in behalf of households,
 while the CEX only measures householdout-of-pocket spending. Figure \ref{fig:nipavcex} shows sector level comparison 
 of Per-capita expenditure in the CEX with corresponding spending in subcomponents of the PCE. 
 Some series such as gasoline and vehicle spending track each other quite closely, while for others 
 (particularly services) spending in the PCE is much higher than spending in the CEX. A gap between the PCE and the 
 CEX should be expected as the CEX includes only household spending, while the PCE includes consumption spending by 
 third-parties (for example employer provided health-insurance). 

Another key difference between the PCE and the CEX, is the CEX is reported directly by households. This introduces two 
issues: (1) reporting smoothing (Add citations) and (2) under-reporting (Aguiar and Bils 2015 and add other citations). Expenditure smoothing
could possibly explain by the fall in expenditure during 2007-09 is larger in the PCE series than in the CEX. Under-reporting appears to 
have grown over-time. 


\begin{figure}
    \caption{NIPA v CEX}
    \label{fig:nipavcex}
    \begin{tabular}{cc}
\includegraphics[scale=.2]{figures/gas_cexvnipa} &
\includegraphics[scale=.2]{figures/food_cexvnipa} \\
\includegraphics[scale=.2]{figures/clothing_cexvnipa} &
\includegraphics[scale=.2]{figures/nondurables_cexvnipa} \\
\includegraphics[scale=.2]{figures/mvparts_cexvnipa} & 
\includegraphics[scale=.2]{figures/furn_cexvnipa} \\
\includegraphics[scale=.2]{figures/durables_cexvnipa} & 
\includegraphics[scale=.2]{figures/services_cexvnipa} \\
    \end{tabular}
\end{figure}

\begin{figure}
    \caption{Total ``PCE'' Consumption Spending}
    \label{fig:total_pce}
    \centering
\includegraphics[scale=.5]{figures/pce_cexvnipa}

\end{figure}



During the period of the 2008 rebates, summer 2008, are there any abnormal changes in the relationship between the PCE series and the CEX series?
Figure \ref{fig:lR_nipavcex} shows the log-difference of expenditure in the PCE with expenditure in the CEX $\log(PCE)_t -\log(CEX)_t$ from 
2007-2010. It does not appear that the difference for the overall PCE or services changes much around the summer of 2008.\footnote{Durables
are not pictured because its series is very volatile.} However, there is a large 
drop in the log-difference for non-durables (this means that reported expenditure  in the CEX rises compared to the PCE), which troughs 
in April 2008 right before the first stimulus checks were sent out. 




%\begin{figure}
 %   \caption{Log-Ratio NIPA and CEX Expenditure 2007-2010}
 %   \label{fig:lR_nipavcex}
 %   \centering
%\includegraphics[scale=.5]{figures/diff_cexvnipa}

%\end{figure}


We formally test whether the PCE/CEX relationship was abnormally different during the summer of 2008 by regressing the log-difference
of the two series on an indicator for May-August of 2008, when X percent of the rebate was distributed:

\begin{align}
    \log \left( \frac{NIPA Series}{CEX Series}   \right)_t = \beta_0 + \beta_1 1(t = \textit{May-August 2008})_t + T_t + \epsilon_t,
\end{align}

where $T_t$ is a linear time-trend meant to compensate for the increase in CEX under-reporting over-time. Table \ref{table:pceVcex}
shows the results of this analysis. Column one shows that even when compensating for the linear time-trend, the gap 
between reported PCE and CEX spending is smaller than average. Column two also adds indicators for the other months
in 2008 (January-April and September-December). The coefficents on these indicators are similar in magnitude to 


[Side-thought using time fixed effects to look at above-below average change in a volatile series is quite problematic.
Why? Just about everything is significantly different from normal when the series is volatile. Better may to take the 
reported fixed effects and then see how much they differ from each other.]


