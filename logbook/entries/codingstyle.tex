\section{Coding Style}

While we all have our own style of codes, there are best practices that we should all adhere to.

Below is a Python code snippet from a project. We use it to illustrate the use of sections, comments, and line length.
\begin{lstlisting}[language=python]
# ----------------------------------------------------------------------
# LOAD INPUT FILE
# ------------------------------------------------------------------------

# function that reads appended fmli files
df = pd.read_parquet('../input/fmli.parquet')

# ------------------------------------------------------------------------
# SET INDEX
# ------------------------------------------------------------------------ 
                   
# set index of dataframe
df.set_index('NEWID', inplace=True) 

# ----------------------------------------------------------------------
# DATA IS SEPARATED BY PQ (PREVIOUS QUARTER) AND CQ (CURRENT QUARTER) BASED
# ON THE INTERVIEW MONTH. THE NEXT STEP SUMS THESE TWO VARIABLES TO
# CALCULATE TOTAL CONSUMPTION OVER PREVIOUS THREE MONTHS FROM INTERVIEW
# DATE
# ------------------------------------------------------------------------

# column names for the variables that end with CQ. these are the expenditure
# values for the previous quarter. 
df_columns_filter = df.filter(regex='CQ$',axis=1).columns

# next we sum previous quarter and current quarter expenditure variables
for column_name in df_columns_filter:

    # remove the CQ subscript from the column name
    expen_column = column_name.replace('CQ','',1)

    # SUM CQ and PQ expenditures
    df[expen_column] = df[expen_column + 'PQ'] + df[expen_column + 'CQ']

    # Drop CQ and PQ columns
    df.drop([expen_column + 'PQ',expen_column + 'CQ'], inplace=True, axis=1)
\end{lstlisting}


\subsection{Sections}

 The code is separated into sections using clear comment dividers. The sections corresponds to broad tasks within the code. The benefits of this organization are:
\begin{enumerate}
	\item It is clear when a certain task begins and end. If we want to debug a certain part of the code it should be clear from the sections where it is located.
	\item The code is much more readable. Rather than parsing an entire document, we can parse one section at a time, make sure we understand it, and move on to the next task.
\end{enumerate}

\subsection{Comments}

Each of the code lines is preceded by a comment line that explains what the code is doing.
\begin{itemize}
	\item Comments of this type should generally be a short generic description.
	\item When comments are too specific they are often not updated as the code is changed. We do not want to create more maintenance work for us.
\end{itemize}

\subsection{Line Length}

Lines should be no longer than 100 characters to facilitate the use Git. To enforce this rule for myself, I have a vertical line at 80 characters in my code editors.

Do not use automatic line breaks or line-wrapper in your editor as these will display as one (very) long line in version control.

\subsection{Forbidden Actions}

Now that we laid out what is good to do, here are bad habits that SHOULD BE AVOIDED AT ALL COSTS.

\begin{itemize}
	\item NEVER copy-paste code. Whenever you think about CTRL/CMD + C, an alarm bell should go off in your head!
	
	``Don't repeat yourself`'' is canonical advice given to all coders.
	
	If similar code needs to run more than once, put it in a function or loop.
	
	Why? A (conservative) rule of thumb is that we spend at least twice as much time maintaining code than writing it. By copying code we double our maintenance cost. That is, if we remember to actually update both parts of the code!
	
	General advice is:
	\begin{itemize}
		\item If you need to change one thing in the code write it as a loop.
		\item If you need to change more than one thing in the code write as a function.
		\item If you need to use the code in more than one file write it as a separate script (ado, py module, m).
	\end{itemize}
	
	\item NEVER hard code values in a function. Parameters, sample selections, etc are arguments.
	
	Best practice is to not include any hard numbers in code files at all. Rather, store them in a separate ``options'' file and import the numbers. This practice ensures that if we change one number (e.g. shorten the sample), we do so for the entire project.
	
	YAML or JSON are good formats for storing numbers. We have a folder called ``globaloptions'' for such files.
	
	\item NEVER use absolute file paths. They are almost guaranteed to break the code on any machine but your own.
\end{itemize}
