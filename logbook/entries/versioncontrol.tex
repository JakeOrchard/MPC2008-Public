\section{Version Control}

Version control means we can edit without fear:
\begin{itemize}
	\item Fear of breaking our own code.
	\item Fear of breaking the code for anyone else.
\end{itemize}
Did you ever spend a day working on code, only to then realize that it broke other parts? And that you now have to work another day to undo your changes to understand what went wrong?

Version control is for you!

The basics of version control are:
\begin{itemize}
	\item You can work on your own distribution of the software without impacting others.
	\item Once you are done run make: If it works, you can commit to the common repository. If not, you must fix your error messages before uploading to the repo.
	\item By pulling from the repo you update your code base with the changes others have made.
\end{itemize}

Why not Dropbox? Michael Stepner: \href{https://michaelstepner.com/blog/git-vs-dropbox/}{Git vs Dropbox}

\subsection{Starting Git}

We use Git for version control.

To get started with working on an \textbf{existing} repository: 
\begin{enumerate}
	\item Navigate to the dropbox folder of the project.
	\item\label{step:folder} Create an empty local folder, in which the code will be stored.
	\item IMPORTANT: Disable Dropbox syncing for this folder.
	\begin{itemize}
		\item By syncing all changes at once across computers Dropbox undoes all the benefits of version controls.
	\end{itemize}
	\item In terminal type \texttt{git clone REMOTE\_GIT\_ADDRESS LOCAL\_FOLDER\_DIRECTORY}.
	\begin{itemize}
		\item \texttt{REMOTE\_GIT\_ADDRESS} will be usually of the form \texttt{https://username@[bitbucket/github].org/jfwieland/repositoryname.git}
		\item The \texttt{LOCAL\_FOLDER\_DIRECTORY} is the directory of the folder you created in step \ref{step:folder}.
	\end{itemize}
\end{enumerate}


%\begin{enumerate}
%	\item On the command line navigate to that folder and type \texttt{git init}.
%	\item Connect to the repo on Bitbucket using \texttt{git remote add origin <host-or-remoteURL>}.
%	\item Type \texttt{git pull}.
%\end{enumerate}



\subsection{Using Git}

For the most part we only need a few commands in our daily use of git:

\begin{lstlisting}[language=bash]
git status
# Gives you an overview of the status of your repository:
# 1. Whether your repository is up to date.
# 2. Files that have changes and files that are staged for commit.
# 3. Files that are not tracked.

git pull
# Updates your local code base with the contents of the remote repository.
# Your code base should be up to date after running this command.

git branch
# List all of the branches in your repository.

git branch <branch>
# Create a new branch called <branch>. This does not check out the new branch.

git branch -d <branch>
# Delete the specified branch. This is a ``safe'' operation in that Git prevents
# you from deleting the branch if it has unmerged changes.

git checkout <existing-branch>
# Check out the specified branch, which should have already been created with
# git branch. This makes <existing-branch> the current branch, and updates the
# working directory to match.

git checkout -b <new-branch>
# Create and check out <new-branch>. The -b option is a convenience flag that
# tells Git to run git branch <new-branch> before running git checkout
# <new-branch>. git checkout -b <new-branch> <existing-branch>

git add <filename>
# Stages files that have been modified for a commit to the remote depository.
# You have to repeat this command for every file you want to add.

git commit -m "<commit message>"
# Commits the files that have been staged (added) to the remote depository.
# The commit message should describe the key objectives and changes that were made.
# You can sequentially commit individual file. 
# Bitbucket bundles these commits and displays the individual commit messages for each part.

\end{lstlisting}

\subsection{Merge Conflicts}

Merge conflicts are not uncommon but often can be resolved without much trouble. You can follow these steps:
\begin{enumerate}
	\item \texttt{git checkout <the branch you have been working on>}
	\item \texttt{git pull origin <master branch on your machine>} (on my machine this is called ``master'').
	\item You should see a message that the merge failed because of file conflicts.
	\item Navigate to the conflicted files in your code editor (VSC, Atom). In the sections with conflicts you should see options to accept either the incoming change or the outgoing change (or both). Select the appropriate one.
	\item Add and commit the resolved files.
	\item \texttt{git push origin <the branch you have been working on>}
\end{enumerate}
