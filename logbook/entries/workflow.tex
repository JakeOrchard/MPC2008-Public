\section{Steps of Our Work Flow}

These are the steps you should always follow. More details on each step is included in the links below.

\begin{enumerate}
	\item Check for task assignments on Asana/Github.
	\item Before starting a task, pull from the repo using \texttt{git pull} and create your own branch using \texttt{git branch -b <name of new branch>}. If you have already created a branch and want to continue working on it, use \texttt{git checkout <name of existing branch>}. 
	\item Work on your task.
	\item After you have completed your task run \texttt{make} in your task directory. If that works, run \texttt{make} in the base directory.
	\item ONLY IF EVERYTHING EXECUTES WITHOUT ERROR: commit your changes and push to the repo.
	\item Write a short summary of your results for the logbook. Include tables and graphs by linking to the output folder of your task.
	\item Record the task in Asana/Github: Describe what you did and briefly what the results look like.
\end{enumerate}

\section{References}

The goal of this workflow is to ensure that our work is reproducible and to minimize our work in maintaining the code base. It is based upon the experiences and solutions documented by others:

\begin{itemize}
	\item Gentzkow and Shapiro: \href{https://web.stanford.edu/~gentzkow/research/CodeAndData.pdf}{``Code and Data for the Social Sciences: A Practitioner's Guide''} and their \href{https://github.com/gslab-econ/ra-manual/wiki/Introduction}{GitHub RA manual}
	\item Patrick Ball: \href{https://hrdag.org/2016/06/14/the-task-is-a-quantum-of-workflow/}{The Task Is A Quantum Of Workflow} and his \href{https://www.youtube.com/watch?v=ZSunU9GQdcI}{associated presentation on youtube}.
	\item Michael Stepner: \href{https://michaelstepner.com/blog/git-vs-dropbox/}{Git vs Dropbox}
\end{itemize} 

The best way to learn the methods we will be employing here is through the core and additional lessons from \href{https://software-carpentry.org/lessons/}{Software Carpentry}. At a minimum you should cover the core lessons in Python and Git. You are also strongly encouraged to work through the lessons on the Unix Shell and Make.







